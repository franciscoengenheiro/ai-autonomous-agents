\chapter{Conclusão}\label{ch:conclusao}

O projeto permitiu adquirir conhecimentos sobre inteligência artificial e os mecanismos base necessários para a implementação de agentes autónomos, através da aplicação prática de conceitos estudados ao longo do semestre.

Outras temáticas, como a arquitetura de software e o processo de desenvolvimento de software, foram também abordadas, aprendidas e aplicadas ao longo do projeto, o que permitiu a aquisição de competências em áreas como a análise, o design e a implementação de sistemas de software.
Foram também adquiridas competências de programação em \textit{java} e \textit{python}, especialmente vocacionadas para a programação orientada a objetos, o que permitiu modular, de forma incremental, os diferentes subsistemas das arquiteturas implementadas, reduzindo a complexidade inerente ao desenvolvimento de sistemas de software.

Em relação às diferentes fases do projeto, na primeira parte, foi desenvolvida uma biblioteca onde foram implementados os mecanismos bases relacionados com os conceitos gerais de inteligência artificial (e.g., agente, ambiente) e um jogo que integra essa biblioteca.
O jogo consistia num ambiente virtual onde a personagem tinha por objectivo registar a presença de animais através de fotografias, simulando a exploração no mundo real.
O agente foi implementado com uma arquitetura reativa com memória, mais concretamente, criou-se uma máquina de estados finita no seu módulo de controlo, que permitia a tomada de decisões com base no estado anterior (passado), em eventos pre-definidos do ambiente e na ação a executar pelo agente no momento (presente).
Com esta arquitetura observou-se que o agente conseguia atingir os objetivos propostos, mas não conseguia reagir a situações inesperadas (i.e., eventos não previstos que pudessem ocorrer no ambiente).
Mais ainda, na caracterização do ambiente, considerou-se que o mesmo era dinâmico e parcialmente observável, o que permitiu concluir que a arquitetura reativa implementada não é a mais adequada para o problema proposto, visto que não consegue reagir a situações inesperadas (e.g., os animais podem fugir).

Na segunda parte do projeto, foi desenvolvido um agente reativo com módulos comportamentais para recolher alvos e evitar obstáculos num ambiente virtual com dimensões fixas.
O agente foi implementado com uma arquitetura reativa simples (i.e., sem memória), onde um comportamento composto (que englobava os subcomportamentos: aproximar alvo, evitar obstáculo, explorar) foi integrado no módulo de controlo.
Este módulo comportamental ficou responsável por processar a informação sensorial do agente e escolher a ação a executar com base nessa informação.
Esta integração permitiu a escolha da ação a executar com base num mecanismo de seleção de ação, mais concretamente, por hierarquia fixa de prioridade.
Com esta arquitetura observou-se que, através da interface gráfica da biblioteca SAE, o agente conseguia atingir os objetivos propostos, mas não de forma ótima (i.e., o caminho mais rápido para recolher todos os alvos, evitando os obstáculos, não era seguido).
Quando se utilizou configurações do ambiente mais complexas, o agente não conseguia atingir os objetivos propostos ou demorava muito tempo a atingi-los.
Tal prova que a arquitetura reativa simples implementada não é a mais adequada para o problema proposto, visto que o agente apenas age com base na informação sensorial do momento, não considerando o passado nem o futuro.

Na terceira parte do projeto, foi desenvolvida uma biblioteca para procuras em espaços de estados, que envolveu aprender conceitos de raciocínio automático e o processo inerente da tomada de decisão, mecanismo de procura e a implementação de diferentes estratégias de procura.
No final da parte 3, em forma de exercício prático, foi modelado um problema concreto de contagem, onde foram aplicadas as diferentes estratégias de procura implementadas na biblioteca desenvolvida e comparados os resultados obtidos.
Este exercício permitiu perceber que a escolha da estratégia de procura ideal dependerá de uma análise cuidada dos fatores associados ao problema em questão, de forma a garantir a resolução eficiente e eficaz do mesmo.

Na quarta e última parte do projeto, foi desenvolvido um agente deliberativo para a procura em espaços de estados com a biblioteca desenvolvida na parte 3; e para a procura com processos de decisão de Markov.
Para tal, foram abordados conceitos de planeamento automático e processos de decisão sequencial, tendo sido utilizado o ambiente formalizado na parte 2 para testar o agente com a arquitetura deliberativa implementada em ambas as vertentes.
Observou-se que ambas as abordagens permitiram ao agente atingir os objetivos propostos, de forma ótima, no entanto, a procura em espaços de estados foi mais eficiente que a procura com processos de decisão de Markov, uma vez que o agente conseguia atingir os objetivos propostos mais rapidamente.
Isto deveu-se ao facto do ambiente ser estático, determinístico e completamente observável, e de não se tirar proveito das vantagens associadas à procura com processos de decisão de Markov que são mais evidentes em ambientes dinâmicos e estocásticos.

Após análise dos resultados obtidos, concluiu-se que apenas o agente deliberativo implementado pode ser considerado um agente relacional (ver secção~\ref{sec:agente-relacional}), uma vez que consegue realizar as ações corretas de forma ótima, e não apenas atingir os objetivos propostos.

 É também importante realçar que, apesar da aprendizagem por reforço ter sido abordada na parte 4, não foi implementada no projeto.
No entanto, permitiu refletir sobre o dilema entre explorar novas ações e aproveitar o conhecimento obtido até ao momento para agir em conformidade com aquilo que se sabe ser o melhor a fazer (como aliás pode ser até extrapolado para a vida real).
Além disso, discutiram-se estratégias para resolver este dilema, despertando o interesse em explorar mais a fundo esta temática no futuro e aplicá-la num contexto concreto.


\section{Trabalho Futuro}\label{sec:trabalho-futuro}

Como trabalho futuro, seria interessante explorar configurações de ambiente não determinísticas e parcialmente observáveis, de forma a perceber como os agentes implementados se comportam nestas condições.
Percebendo assim, a real importância da arquitetura deliberativa com processos de decisão de Markov.

Estudando a biblioteca SAE, providenciada para a segunda parte do projeto, seria interessante perceber como é que se poderia adicionar uma opção para que os obstáculos se mexessem ou existisse uma mudança de configuração a cada x tempo, para poder observar como é que o agente se adaptava.
Também teria que se rever a implementação, no controlo deliberativo, do modelo do mundo, porque o sinal que é dado que o estado do mundo mudou apenas tem em conta quando foi recolhido um alvo no ambiente.
