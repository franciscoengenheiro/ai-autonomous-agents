\chapter{Conclusão}\label{ch:conclusao}

O projeto permitiu adquirir conhecimentos sobre inteligência artificial e os mecanismos base necessários para a implementação de agentes autónomos, através da aplicação prática de conceitos estudados ao longo do semestre.

Outras temáticas, como a arquitetura de software e o processo de desenvolvimento de software, foram também abordadas, aprendidas e aplicadas ao longo do projeto, o que permitiu a aquisição de competências em áreas como a análise, o design e a implementação de sistemas de software.

Foram adquiridas competências de programação em \textit{java} e \textit{python}, especialmente vocacionadas para a programação orientada a objetos, o que permitiu modular, de forma incremental, os diferentes subsistemas das arquiteturas implementadas, reduzindo a complexidade inerente ao desenvolvimento de sistemas de software.

Em relação as diferentes partes do projeto, na primeira parte, foi desenvolvida uma biblioteca em \textit{java} onde foram implementados os mecanismos bases relacionados com os conceitos gerais de inteligência artificial (e.g., agente, ambiente) e um jogo que integra essa biblioteca.
O jogo consistia num ambiente virtual onde a personagem tinha por objectivo registar a presença de animais através de fotografias, simulando a exploração no mundo real.
O agente foi implementado com uma arquitetura reativa com memória, mais concretamente, criou-se uma máquina de estados finita no seu módulo de controlo, que permitia a tomada de decisões com base no estado anterior, em eventos pre-definidos do ambiente e na ação a executar pelo agente.
Com esta arquitetura observou-se que o agente conseguia atingir os objetivos propostos, mas não conseguia reagir a situações inesperadas (i.e., eventos não previstos que pudessem ocorrer no ambiente).
Mais ainda, na caracterização do ambiente, considerou-se que o mesmo era dinâmico e parcialmente observável, o que permite concluir que a arquitetura reativa implementada não é a mais adequada para o problema proposto, visto que não consegue reagir a situações inesperadas.

Na segunda parte do projeto, foi desenvolvido um agente reativo em \textit{python} com módulos comportamentais para recolher alvos e evitar obstáculos num ambiente virtual com dimensões fixas.
O agente foi implementado com uma arquitetura reativa simples (i.e., sem memória), onde um comportamento composto (que englobava os subcomportamentos: aproximar alvo, evitar obstáculo, explorar) foi integrado no módulo de controlo.
Este módulo comportamental ficou responsável por processar a informação sensorial do agente e escolher a ação a executar com base nessa informação.
Esta integração permitiu a escolha da ação a executar com base num mecanismo de seleção de ação, mais concretamente, por hierarquia fixa de prioridade.
Com esta arquitetura observou-se, através da interface gráfica da biblioteca SAE, que o agente conseguia atingir os objetivos propostos, mas não de forma ótima (i.e., o caminho mais rápido para recolher todos os alvos, evitando os obstáculos, não era seguido).

Após análise dos resultados obtidos, concluiu-se que nenhum dos agentes implementados pode ser considerado um agente relacional (ver secção~\ref{sec:agente-relacional}), uma vez que não conseguem realizar as ações corretas de forma ótima, apenas conseguem atingir os objetivos propostos. Visto que ambas as arquiteturas implementadas são arquiteturas reativas, a resposta pode estar nas arquiteturas deliberativas e híbridas (ver secção~\ref{sec:arquiteturas-agente}), que não foram abordadas neste projeto.


FALAR SOBRE O METODO DE ALTERADO NO MODELO DO MUNDO
