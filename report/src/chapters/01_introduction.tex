\chapter{Introdução}\label{ch:introducao}

Este relatório documenta o projeto desenvolvido no âmbito da unidade curricular de Inteligência Artificial para Sistemas Autónomos, da Licenciatura em Engenharia Informática e de Computadores.

O projeto tem como objetivo a aprendizagem de conceitos de inteligência artificial e a sua aplicação no desenvolvimento de sistemas autónomos inteligentes.
O capítulo~\ref{ch:enquadramento-teorico} descreve esses conceitos que servem de base aos restantes temas estudados e ao projeto realizado.

O projeto está dividido em duas partes:
\begin{itemize}
    \item \textbf{Parte 1}: Desenvolvimento de uma biblioteca em Java para a implementação de agentes autónomos inteligentes; Desenvolvimento de um jogo em Java que integra essa biblioteca (ver capítulo~\ref{ch:projeto-parte1}).
    \item \textbf{Parte 2}: Desenvolvimento de um agente reativo em Python com módulos comportamentais para recolher alvos e evitar obstáculos num ambiente com dimensões fixas (ver capítulo~\ref{ch:projeto-parte2}).
\end{itemize}

Cada parte do projeto está organizada de forma a descrever o objetivo principal no contexto geral do projeto; fazer uma síntese dos conceitos estudados que se acharam relevantes para a sua implementação; caracterização do ambiente; descrever a implementação realizada, nomeadamente, a arquitetura do agente implementada; justificar as principais decisões tomadas e a sua relação com os conceitos estudados; e apresentar a estrutura do código desenvolvido.

Existe um capítulo dedicado à revisão do projeto realizado, onde são identificados e descritos os erros cometidos nas entregas realizadas, com a indicação do problema e da respetiva correção, justificada com base nos conhecimentos adquiridos.

O projeto foi realizado individualmente e as entregas foram realizadas semanalmente, de forma incremental, de acordo com o plano de trabalho definido.
