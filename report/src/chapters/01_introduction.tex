\chapter{Introdução}\label{ch:introducao}

Este relatório documenta o projeto desenvolvido no âmbito da unidade curricular de Inteligência Artificial para Sistemas Autónomos (IASA), da Licenciatura em Engenharia Informática e de Computadores (LEIC).

O projeto tem como objetivo a aprendizagem de conceitos de inteligência artificial e a sua aplicação no desenvolvimento de sistemas autónomos.

\section{Organização do Documento}\label{sec:organizacao-documento}

O presente documento está organizado da seguinte forma:
\begin{itemize}
    \item \textbf{Enquadramento Teórico}: Descrição dos conceitos de inteligência artificial e de sistemas autónomos, que servem de base ao projeto realizado;
    \item \textbf{Projeto}: Descrição do projeto desenvolvido, dividido em quatro partes, cada uma com um objetivo específico:
    \begin{itemize}
        \item \textbf{Parte 1}: Desenvolvimento de uma biblioteca em \textit{java} que fornece os mecanismos base para a implementação dos subsistemas que representam os conceitos gerais de inteligência artificial (e.g., agente, ambiente) e outros conceitos relacionados (e.g., máquina de estados); Desenvolvimento de um jogo que integra essa biblioteca (ver capítulo~\ref{ch:projeto-parte1});
        \item \textbf{Parte 2}: Desenvolvimento de um agente reativo em \textit{python} com módulos comportamentais para recolher alvos e evitar obstáculos num ambiente com dimensões fixas (ver capítulo~\ref{ch:projeto-parte2}).
        \item \textbf{Parte 3}: Desenvolvimento de uma biblioteca para procuras em espaços de estados (PEE) em \textit{python} que envolveu a implementação de diferentes estratégias de procura; Desenvolvimento de uma biblioteca para modelação de problemas de procura em \textit{python} que permite a definição de problemas concretos de forma independente da estratégia de procura a aplicar; Modelação de um problema concreto de procura (ver capítulo~\ref{ch:projeto-parte3}).
    \end{itemize}
\end{itemize}

Cada parte do projeto está organizada de forma a descrever o objetivo principal no contexto geral do projeto; fazer uma síntese dos conceitos estudados que se acharam relevantes para a sua implementação; caracterização do ambiente onde o agente vai atuar; descrever a implementação realizada, nomeadamente, a arquitetura do agente usada; justificar as principais decisões tomadas e a sua relação com os conceitos estudados; e apresentar a estrutura do código desenvolvido.

Existe um capítulo dedicado à revisão do projeto realizado, onde são identificados e descritos os erros cometidos nas entregas realizadas, com a indicação do problema e da respetiva correção, justificada com base nos conhecimentos adquiridos.

O projeto foi realizado individualmente e as entregas foram realizadas semanalmente, de forma incremental, de acordo com o plano de trabalho definido.
