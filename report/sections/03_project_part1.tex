\subsection{Arquitetura de software}\label{subsec:arquitetura-de-software}

A arquitetura de software aborda a complexidade inerente ao desenvolvimento de software por meio de uma série de vertentes que estão interligadas. 

\subsubsection{Métricas}\label{subsubsec:acoplamento}

As métricas são medidas de quantificação da arquitectura de um software indicadoras da qualidade dessa arquitectura;

O acomplamento é uma métrica inter-modular que mede o grau de interdependência entre os módulos de um sistema. Pode ser medido através da:
\begin{itemize}[topsep=0pt,itemsep=0pt,partopsep=0pt, parsep=0pt]
    \item \tb{Direção}: Unidirecional vs Bidirecional (uni representa menos acoplamento);
    \item \tb{Visibilidade}: Quando menor for a visibilidade de um módulo, menor é o  seu acoplamento;
    \item \tb{Ordem}: (de menos acoplamento para mais) Herança $\rightarrow$ Composição $\rightarrow$ Agregação $\rightarrow$ Associação $\rightarrow$ Dependência.
\end{itemize}

A coesão é uma métrica intra-modular que determina o nível de coerência funcional de um subsistema/módulo, seja pela sua organização (i.e., cada modulo está organizado por conteúdo) ou pela sua funcionalidade (e.g., \ti{single responsibility principle} - cada modulo tem uma única responsabilidade).

\subsubsection{Princípios}\label{subsubsec:principios}

Os princípios no contexto da arquitectura de software são um conjunto de convenções que orientam a sua definição, garantindo a qualidade de produção da mesma. Alguns exemplos são:
\begin{itemize}[topsep=0pt,itemsep=0pt,partopsep=0pt, parsep=0pt]
    \item \tb{Abstração}: Define a forma como os componentes de um sistema são representados, permitindo a ocultação de detalhes de implementação;
    \item \tb{Modularização}: Ao qual está associado a decomposição (e.g, divisão do sistema em sub-módulos) e o encapsulamento (i.e., ocultação de detalhes de implementação e/ou manutenção de estado privado e interno);
    \item \tb{Factorização}: Onde a arquitectura é dividida em camadas, cada uma com um conjunto de responsabilidades bem definidas. Pode ser estrutural (e.g, Herança) e Funcional (e.g, Delegação);
\end{itemize}
    
\subsection{Processo de Desenvolvimento de Software}\label{subsec:processo-de-desenvolvimento-de-software}

O processo de desenvolvimento de software consiste na
criação da organização de um sistema de forma
progressiva, através de diferentes níveis de abstracção:

\begin{itemize}[topsep=0pt,itemsep=0pt,partopsep=0pt, parsep=0pt]
    \item \tb{Modelo (Conceptual)}: Representação abstrata do sistema, que define o que o sistema deve fazer, sem especificar como;
    \item \tb{Arquitetura (Modelo Concreto)}: Representação concreta do sistema, que define como o sistema deve ser implementado;
    \item \tb{Implementação}: Código fonte que implementa o sistema definindo como o sistema deve ser executado.
\end{itemize}

Consiste num processo iterativo, em que as diferentes actividades de desenvolvimento são alternadas ao longo do tempo em função do conhecimento e do nível de detalhe
envolvido. Essa alternância poderá ser circular (i.e., implementação $\rightarrow$ arquitectura $\rightarrow$ modelo $\rightarrow$ implementação).

\begin{table}
    \centering
    \caption{Tipos de Implementação}
    \label{tab:tipos-de-implementacao}
    \begin{tabular}{|l|l|p{8cm}|}
        \hline
        \tb{Tipo}                      & \tb{Modelo Associado}  & \tb{Designação} \\ \hline
        Estrutural & \ti{UML} & Define a estrutura de um sistema, ou seja, a forma como os componentes se relacionam entre si. \\ \hline
        Comportamental & \ti{Sequence Diagram} & Define o comportamento de um sistema, ou seja, a forma como os componentes interagem e comunicam entre si. \\ \hline
    \end{tabular}
\end{table}

Os diagramas de sequência ou atividade representam o fluxo de controlo de um sistema, ou seja, a sequência de atividades que um sistema executa e a sua ordem. Definem-se como modelos de interação com uma organização bidirecional (i.e., horizontal $\rightarrow$ tempo e vertical $\rightarrow$ estrutura) e são compostos por diferentes elementos de modelação (e.g., mensagens, operadores, linha de vida).

A linguagem de modelação unificada (UML) representa um modelo de comportamento com interação como perspetiva principal de modelação. Este tipo de modelação descreve a forma como as partes de um sistema interagem entre si e com o exterior para produzir o comportamento do sistema.
