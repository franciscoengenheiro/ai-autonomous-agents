\subsection{Necessidade de Desenho de Software Resiliente}\label{subsec:necessidade-de-desenho-de-software-resiliente}
Os sistemas distribuídos representam um conjunto de computadores independentes e interligados em rede, que se apresentam aos utilizadores como um sistema único e coerente~\cite{fcc-distributed-systems}.

Dado a constante necessidade destes sistemas estarem disponíveis, aliados à sua complexidade de funcionamento, é natural que estejam suscetíveis a falhas de comunicação, de hardware, de software, entre outras.
Por esse motivo, existe a necessidade de garantir que os serviços que disponibilizam sejam resilientes, e mais concretamente, tolerantes a falhas.

Um serviço tolerante a falhas, é um serviço que é capaz de manter a sua funcionalidade total ou parcial, ou apresentar uma alternativa, quando um ou mais componentes que lhes estão associados falham.
De forma a alcançar este objetivo, foram desenhados mecanismos de resiliência.
Alguns exemplos:

\begin{itemize}[topsep=0pt,itemsep=0pt,partopsep=0pt, parsep=0pt]
    \item \textbf{Retry}: Tenta novamente uma operação que falhou, aumentando a sua probabilidade de sucesso;
    \item \textbf{Rate Limiter}: Limita a taxa de requisições que um determinado serviço pode receber;
    \item \textbf{Circuit Breaker}: Interrompe, temporariamente, a comunicação com um serviço que está a falhar, de forma a evitar que o mesmo sobrecarregue o sistema. Semelhante a um disjuntor elétrico;
    \item \textbf{Fallback}: Fornece um valor ou executa uma ação alternativa caso uma operação falhe.
\end{itemize}

\subsection{Mecanismos de Resiliência}\label{subsec:bibliotecas-que-fornecem-mecanismos-de-resiliencia}

Existem bibliotecas que fornecem mecanismos de resiliência (Tabela~\ref{tab:resilience_libraries}).
Estes atuam em tempo de execução e implementam uma determinada estratégia.
A configuração de um mecanismo de resiliência é feita através de uma política que define o seu comportamento.

\begin{table}[h]
    \centering
    \caption{Exemplos de bibliotecas que fornecem mecanismos de resiliência.}
    \label{tab:resilience_libraries}
    \begin{tabular}{|l|l|l|l|}
        \hline
        \textbf{Biblioteca}                      & \textbf{Linguagem} & \textbf{Plataforma} \\ \hline
        Netflix's Hystrix~\cite{netflix-hystrix} & Java               & JVM                 \\ \hline
        Resilience4j~\cite{resilience4j}         & Java/Kotlin        & JVM                 \\ \hline
        Polly ~\cite{polly-dotnet}               & C\#                & .NET                \\
        \hline
    \end{tabular}
\end{table}

A biblioteca \textit{Polly}~\cite{polly-dotnet} divide os mecanismos de resiliência em duas categorias:

\begin{itemize}[topsep=0pt,itemsep=0pt,partopsep=0pt, parsep=0pt]
    \item \textbf{Resiliência Reativa}: Reage a falhas e mitiga o seu impacto (e.g., \textit{Retry, Circuit Breaker});
    \item \textbf{Resiliência Proativa}: Previne que as falhas aconteçam (e.g., \textit{Rate Limiter, Timeout}).
\end{itemize}
